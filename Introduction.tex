DNA microarrays since its first introduction has become a very promising biotechnology which is one of the most powerful and widely used technology. Microarrays were being applied almost everywhere in life sciences, both in molecular biology and in medicine to address a wide range of problems, from bacterial infections to the classification of tumors \citep{Boldrick:2002p2425, Golub:1999p14}, even from the detection of alternative splicing to the hybridization of comparative genomes. 
\\
With the power of measuring the mRNA abundance at the genomic level, the identifying of differential expressed genes become the most interested question in microarray data analysis, that is, the question which genes have the expression levels are changed during important biological processes (e.g. cellular replication (time), treatment/control cell type, dose of a drug), or varied across collections of related samples (e.g. tumor samples from patients with cancer, from patients after a specify drug treatment).
\\
The biological question of which gene is differential expressed can be formulated as the statistical test of the null hypothesis: there's no association between the expression levels of the gene and the responses.
\\
It seems simple with many powerful techniques from statistic, e.g. Student's t-test, but many problems arisen from the fact that DNA microarrays experiments deliver ton of informations. One of the problems is the multiple testing problem was considered in \citep{Dudoit:2003p3}. We consider in this paper another problem, well, let cast aside the multiple problem, we assume that most of classical statistical tests perform well to identify the genes, assumed that we have enough samples, in other words, a huge dataset is what we need to perform those tasks.
At the first sight, all of microarray datas are huge, but unfortunately, it is huge because of the large number of genes represented in their array (The human genome has approximately about 23.000 genes, most of model organisms was used in research like Arabidopsis thaliana has around 25.000 genes, some higher plants have even more genes, the fewest lies around at thousand genes, but it's still a huge number).
\\
Typically because of the costly production of microarray, for each gene the number of RNA samples assayed is small. Therefore, the commonly used approach of treating the differential expression of one gene at a time as a solely population often has low power. The Student's t-test have been shown with very low power when the samples size falls under 10. Some could consider the assumption that all genes have the same variance (to stabilize the variance estimator, but in the practice often not this case) to increase the power of detection but also increase the risk of generating false detection if the assumption is false. More complicated, the measured datas are often non-normally distributed and have non-identical and dependent distributions between genes. 
\\
Because of those problems, in the last few years various advanced approaches have been suggested, from modifying the simple fold change to the classical ordinary t statistic. The idea of modifying the classical fold change method \citep{Kadota:2008p275} bases simply on empirical observation from real data, although it doesn't base on any statistical concept, surprisingly performs well with real datas. The modification of Student's t-test based mostly on modifying the estimators of variance. The widely used SAM t-test \citep{Tusher:2001p295} for example adds a small constant to the gene-specific variance estimator to stabilize the small variance. Another used the empirical Bayesian approaches to have a more stabilized estimator of the variance. One important point to mention here is, the parallel nature of the inference in microarrays (variation because of technical noises) and in biology (genes in the same tissues tend to have same expression profile) allows some possibilities for borrowing information from the ensemble of genes which can assist in inference about each gene individually. That "prior" knowledge about microarray should be considered in any Bayesian approaches.
\\
The advantage of Bayesian approaches is, because they allow the information sharing between genes, which is essential when the number of sample is small. The disadvantage of the Bayesian approaches is that they could become quite computationally and analytically expensive, even if the assumption of normally distributed for the data is true. One of the approaches has been suggest recently was from \citep{OpgenRhein:2007p11}. This "shrinkage" approach was shown to be simple as the SAM test but performs well like any another full Bayesian approaches, yet even could be derived fully analytic, requires no computer-intensive procedures, and makes no prior assumptions about the distribution of data.
\\
We know that most of microarrays experiments focus on the two factorial question, means just two responses are compared, e.g. one need to compare tumor cell samples with tumor cell after a drug treatment samples to detect which genes are differential expressed after the treatment. Sometimes but more than two factors are concerned, e.g. the drug treatment after one specific time interval (24h, 48h), or the same tumor cell but with various drugs. In this case the simple fold change fails, but what happens with the well-known t-test? The t-test appears at the first look just an extent of fold change with the consideration of the variance. Fortunately, another well-known extension of the Student t-test, F-test or ANOVA ({\bf AN}alysis {\bf O}f {\bf VA}riance) can handle any desirable factorial questions.
\\[1ex]
The purpose of this paper is further examination of the "good" shrinkage approach when applied in practice. First we shall give the readers a fundamental understanding in microarrays technology, which quite useful for readers who are new to microarray. Then a briefly review of Student t-test and understand the original shrinkage idea of James \& Stein, which crucial to understand our work in this paper. Third is our simulation setup to test the theory of the shrinkage approach in the practice.